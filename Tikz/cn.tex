\RequirePackage{luatex85}
\documentclass[border=1mm]{standalone}
\usepackage{myFig}
\begin{document}
	\begin{tikzpicture}
		\newcommand\para[3]{
			\coordinate (a) at #1;
			\coordinate (b) at ($(a) + #2$);
			\coordinate (c) at ($(a) + #3$);
			\coordinate (d) at ($(c) + #2$);
			\draw (a) -- (b) -- (d) -- (c) -- cycle;
		}
		\newcommand\fillpara[4]{
			\coordinate (a) at #1;
			\coordinate (b) at ($(a) + #2$);
			\coordinate (c) at ($(a) + #3$);
			\coordinate (d) at ($(c) + #2$);
			\draw[fill=#4] (a) -- (b) -- (d) -- (c) -- cycle;
		}
		
		\para{(0, 1.5)}{(1.7, 0)}{(0.5, 0.4)}{white}

		% 基盤の絵を描く
		\para{(0, 0)}{(1.7, 0)}{(0.5, 0.4)}
		\foreach \x in {0.3,0.6,...,1.8} {
			\para{(\x, 0.1)}{(0.1, 0)}{(0.1, 0.08)}
			\para{(\x+0.20, 0.25)}{(0.1, 0)}{(0.1, 0.08)}
		}
		
		% 下側の線
		\foreach \x in {0.3,0.6,...,1.8} {
			\draw(\x+0.05, 0.1) ++(0, 0.5) 
				++(0.05, 0.05) -- ++(0, 0.5);
			\draw(\x+0.25, 0.3) ++(0, 0.5)
				++(0.05, 0.05) -- ++(0, 0.5);
		}
		
		\fill[white](0.8, -0.25) rectangle ++(0.25, 0.30);
		
		\fillpara{(0, 1)}{(1.7, 0)}{(0.5, 0.4)}{white}
		\fill[white](0.8, 1.4) rectangle ++(0.25, 0.25);
		
		% 上側の線
		\foreach \x in {0.3,0.6,...,1.8} {
			\draw(\x+0.05, 0.1) ++(0, 1.0) 
				++(0.05, 0.05) -- ++(0, 0.5);
			\draw(\x+0.25, 0.3) ++(0, 1.0)
				++(0.05, 0.05) -- ++(0, 0.5);
		}
		
		\fillpara{(1.7, 1)}{(0, 0.5)}{(0.5, 0.4)}{white}

		\draw[fill=white] (0, 1) -- ++(0, 0.5)
		-- ++(0.8, 0) -- ++(0, -0.3) -- ++(0.25, 0)
		-- ++(0, 0.3) -- ++(0.65, 0) -- ++(0, -0.5) -- cycle;

		\draw(0.25, 1.5) -- ++(0.1, -0.15) -- ++(0.1, 0.15);

		\node[rotate=140, scale=0.5] at (0.15, -0.15){1};
		
		\node[scale=0.5, text width=2.5cm, anchor=west]
			(n1) at (-2.25, 1.5)
			{1番ピンを示す\\ △マークが\\ 入っている.};
		\node[scale=0.5, text width=2.5cm, anchor=west]
			(n2) at (-2.25, 0.75)
			{基板には番号が\\ 書いてある.};
		\node[scale=0.5, text width=3.5cm, anchor=west]
			(n3) at (-2.25, 0)
			{基板の1番に\\ △マークを合わせる.};
		\node[scale=0.5, text width=2.5cm] (n4) at (2, -0.35)
			{切り欠きを\\合わせても良い.};

		\draw[-latex] (n1.east) -- ++(0.75, -0.15);
		\draw[-latex] (n2.east) -- ++(1.0, -0.5);
		\draw[-latex] (n4.west) -- ++(-0.40, 0) -| ++(0, 0.30);
		

	\end{tikzpicture}
\end{document}
