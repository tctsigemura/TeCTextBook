%\documentclass[dvipdfmx]{beamer}      % platex の場合
\documentclass{beamer}                 % lualatex の場合
\usepackage{mySld}
\usepackage{multicol}

\begin{document}
\title{基礎コンピュータ工学\\第5章 機械語プログラミング\\(パート6)}
\date{}

\begin{frame}
  \titlepage
\end{frame}

%==============================================================================
%\begin{frame}
%  \frametitle
%  \tableofcontents
%\end{frame}

\section{条件判断の演習}
%==============================================================================
\begin{frame}
  \frametitle{条件判断の演習}
  \begin{enumerate}
  \item[1.] プログラムの作成手順を再度確認
    \begin{enumerate}
    \item[(1)] フローチャートを描く.
    \item[(2)] フローチャートを基にニーモニックを書く.
    \item[(3)] アドレスを決める.
    \item[(4)] 機械語を作る.
    \end{enumerate}
    \vfill
  \item[2.] 演習
    \begin{enumerate}
    \item[(1)] N番地の値がゼロならM番地にゼロを,
      そうでなければM番地に1を格納するプログラム
      \begin{itemize}
      \item LD命令はフラグを変化させないので...
      \item 前回の「条件判断2」のパターンを利用
      \end{itemize}
      \vfill
    \item[(2)] N番地の値とM番地の値で,
      大きい方をL番地に格納するプログラム
      \begin{itemize}
      \item 値は符号付きの数値とする.
      \item 比較は引き算でできる.
      \end{itemize}
    \end{enumerate}
  \end{enumerate}
  \vfill
\end{frame}

\end{document}
