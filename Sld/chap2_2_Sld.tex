%\documentclass[dvipdfmx]{beamer}      % platex の場合
\documentclass{beamer}                 % lualatex の場合
\usepackage{mySld}

\begin{document}
\title{基礎コンピュータ工学\\第2章 情報の表現\\(パート2)}
\date{}

\begin{frame}
  \titlepage
\end{frame}

%==============================================================================
%\begin{frame}
%  \frametitle{}
%\end{frame}

\section{情報の表現}
%==============================================================================
\begin{frame}
  \frametitle{2進数から10進数への変換}
  2進数の桁ごとの重みは,桁の番号をnとすると$2^n$になる.

\[
\begin{array}{c c c c c }
b_3     & b_2     & b_1     & b_0     \\
2^3 = 8 & 2^2 = 4 & 2^1 = 2 & 2^0 = 1 \\
\end{array}
\]

2進数の数値は,その桁の重みと桁の値を掛け合わせたものの合計.\\
例えば2進数の$1010_2$は,$2^3$の桁が1,$2^2$の桁が0,
$2^1$の桁が1,$2^0$の桁が0ですから,
次のように計算できる.

\begin{align*}
1010_2 &= 2^3 \times 1 + 2^2 \times 0 + 2^1 \times 1 + 2^0 \times 0 \\
       &= 8 \times 1 + 4 \times 0 + 2 \times 1 + 1 \times 0 \\
       &= 8 + 0 + 2 + 0 \\
       &= 10_{10}
\end{align*}
\end{frame}

%==============================================================================
\begin{frame}
  \frametitle{2進数から10進数への変換(問題)}

\emph{問題1:}次の2進数を10進数に変換しなさい.
\begin{enumerate}
\item[1)] $0001 1100_2$
\vfill
\item[2)] $0011 1000_2$
\vfill
\item[3)] $1110 0000_2$
\vfill
\end{enumerate}
\end{frame}

%==============================================================================
\begin{frame}
  \frametitle{10進数から2進数への変換}

  2進数を2で割ると右に1桁移動する.\\
  (10進数は10で割ると右に1桁移動した.)\\
  その時の余りは最下位の桁からはみ出した数になる.\\
  同じ値の10進数を2で割っても余りは同じ.

\begin{center}
\begin{tabular}{l r l l }
          & $10_{10}$ & = & $1010_2$                       \\
$\div 2$↓ &                   &                            \\
          & $5_{10}^{\cdots 0}$  & = & $0101_2^{\cdots 0}$ \\
$\div 2$↓ &                    &                           \\
          & $2_{10}^{\cdots 1}$  & = & $0010_2^{\cdots 1}$ \\
$\div 2$↓ &                    &                           \\
          & $1_{10}^{\cdots 0}$  & = & $0001_2^{\cdots 0}$ \\
$\div 2$↓ &                    &                           \\
          & $0_{10}^{\cdots 1}$  & = & $0000_2^{\cdots 1}$ \\
\end{tabular}
\end{center}
\end{frame}

%==============================================================================
\begin{frame}
  \frametitle{10進数から2進数への変換}
2で割る操作を繰り返しながらはみ出して来た数を記録する.\\
右から並べると2進数で表したときの0/1の並びが分かる.

\begin{center}
\begin{minipage}{0.5\columnwidth}
\begin{flushright}
\begin{tabular}{l}
$2 \underline{) ~~10 } $\\
$2 \underline{) ~~~5  } {\cdots 0}$ \\
$2 \underline{) ~~~2  } {\cdots 1}$ \\
$2 \underline{) ~~~1  } {\cdots 0}$ \\
$~~            ~~~~0    {\cdots 1}$
\end{tabular}
\end{flushright}
\end{minipage}
\begin{minipage}{0.4\columnwidth}
余りを右から順に並べると $1010_2$
\end{minipage}
\end{center}
\end{frame}

%==============================================================================
\begin{frame}
  \frametitle{10進数から2進数への変換(問題)}

\emph{問題2:}次の10進数を8桁の2進数に変換しなさい.
\begin{enumerate}
\item[1)] $16_{10}$
\vfill
\item[2)] $50_{10}$
\vfill
\item[3)] $100_{10}$
\vfill
\item[4)] $127_{10}$
\vfill
\item[5)] $130_{10}$
\vfill
\end{enumerate}
\end{frame}

%==============================================================================
\begin{frame}
  \frametitle{16進数}
\begin{minipage}{0.5\columnwidth}
\begin{itemize}
\item 2進数4桁を16進数1桁で書く.
\item 16種類の数字が必要,
\item AからFを数字の代用にする.
\item 2進数の書き方\\
  $01100100_2$\\
  $01100100_b$
\item 16進数の書き方\\
  $64_{16}$\\
  $64H$
\item 右の表は暗記すること.
\end{itemize}
\end{minipage}
\begin{minipage}{0.45\columnwidth}
\centerline{\small\begin{tabular}{ c | c | c }
\hline
\hline
{\bf 2進数} & {\bf 16進数} & {\bf 10進数} \\
\hline
$0000_2$ & $0_{16}$ & $0_{10}$ \\
$0001_2$ & $1_{16}$ & $1_{10}$ \\
$0010_2$ & $2_{16}$ & $2_{10}$ \\
$0011_2$ & $3_{16}$ & $3_{10}$ \\
$0100_2$ & $4_{16}$ & $4_{10}$ \\
$0101_2$ & $5_{16}$ & $5_{10}$ \\
$0110_2$ & $6_{16}$ & $6_{10}$ \\
$0111_2$ & $7_{16}$ & $7_{10}$ \\
$1000_2$ & $8_{16}$ & $8_{10}$ \\
$1001_2$ & $9_{16}$ & $9_{10}$ \\
$1010_2$ & $A_{16}$ & $10_{10}$ \\
$1011_2$ & $B_{16}$ & $11_{10}$ \\
$1100_2$ & $C_{16}$ & $12_{10}$ \\
$1101_2$ & $D_{16}$ & $13_{10}$ \\
$1110_2$ & $E_{16}$ & $14_{10}$ \\
$1111_2$ & $F_{16}$ & $15_{10}$ \\
\end{tabular}}
\end{minipage}

\end{frame}

%==============================================================================
\begin{frame}[fragile]
  \frametitle{16進数のFFまで数えてみよう}
  \begin{lstlisting}[basicstyle={\ttfamily},frame=none]
 00 10 20 30 40 50 60 70 80 90 A0 B0 C0 D0 E0 F0
 01 11 21 31 41 51 61 71 81 91 A1 B1 C1 D1 E1 F1
 02 12 22 32 42 52 62 72 82 92 A2 B2 C2 D2 E2 F2
 03 13 23 33 43 53 63 73 83 93 A3 B3 C3 D3 E3 F3
 04 14 24 34 44 54 64 74 84 94 A4 B4 C4 D4 E4 F4
 05 15 25 35 45 55 65 75 85 95 A5 B5 C5 D5 E5 F5
 06 16 26 36 46 56 66 76 86 96 A6 B6 C6 D6 E6 F6
 07 17 27 37 47 57 67 77 87 97 A7 B7 C7 D7 E7 F7
 08 18 28 38 48 58 68 78 88 98 A8 B8 C8 D8 E8 F8
 09 19 29 39 49 59 69 79 89 99 A9 B9 C9 D9 E9 F9
 0A 1A 2A 3A 4A 5A 6A 7A 8A 9A AA BA CA DA EA FA
 0B 1B 2B 3B 4B 5B 6B 7B 8B 9B AB BB CB DB EB FB
 0C 1C 2C 3C 4C 5C 6C 7C 8C 9C AC BC CC DC EC FC
 0D 1D 2D 3D 4D 5D 6D 7D 8D 9D AD BD CD DD ED FD
 0E 1E 2E 3E 4E 5E 6E 7E 8E 9E AE BE CE DE EE FE
 0F 1F 2F 3F 4F 5F 6F 7F 8F 9F AF BF CF DF EF FF
  \end{lstlisting}

\end{frame}

%==============================================================================
\begin{frame}
  \frametitle{16進数との変換}

  \begin{itemize}
  \item 2進数 <==> 16進数 \\
    4桁の2進数と1桁の16進数の対応は暗記する.
  \item 10進数 <==> 16進数 \\
    \begin{itemize}
    \item 10進数 <==> 2進数 <==> 16進数 \\
      一度,2進数に変換してから変換する.\\
      $100_{10} = 0110 0100_2 = 64_{16}$
    \item 直接計算する \\
      桁の重みは16倍になっていく.
      \[
        \begin{array}{c c c c c }
          h_3     & h_2     & h_1     & h_0     \\
          16^3 = 4096 & 16^2 = 256 & 16^1 = 16 & 16^0 = 1 \\
          2^{12} = 4096 & 2^8 = 256 & 2^4 = 16 & 2^0 = 1 \\
        \end{array}
        \]
    \end{itemize}
  \end{itemize}
\end{frame}

%==============================================================================
\begin{frame}
  \frametitle{16進数との変換}

  \begin{itembox}[l]{10進数 => 16進数}
    \begin{minipage}{0.3\columnwidth}
      \begin{flushright}
        \begin{tabular}{l}
          $16 \underline{) ~~100 } $\\
          $16 \underline{) ~~~~6 } {\cdots 4}$ \\
          $~~~            ~~~~~0   {\cdots 6}$
        \end{tabular}
      \end{flushright}
    \end{minipage}
    \begin{minipage}{0.5\columnwidth}
      余りを右から順に並べると $64_{16}$
    \end{minipage}
  \end{itembox}

  \begin{itembox}[l]{16進数 => 10進数}
    16進数の数値は,その桁の重みと桁の値を掛け合わせたものの合計.
    \begin{align*}
      64_{16} &= 16^1 \times 6 + 16^0 \times 4 \\
      &= 16 \times 6 + 1 \times 4 \\
      &= 96 + 4 \\
      &= 100_{10}
    \end{align*}
  \end{itembox}
\end{frame}

%==============================================================================
\begin{frame}
  \frametitle{16進数(問題1/4)}
\emph{問題3:}$00_{16}$から$FF_{16}$まで,声に出して数えなさい.


\emph{問題4:}次の2進数を2桁の16進数に変換しなさい.
\begin{enumerate}
\item[1)] $0001 1100_2$
\vfill
\item[2)] $0011 1000_2$
\vfill
\item[3)] $1110 0000_2$
\vfill
\item[4)] $0111 0101_2$
\vfill
\end{enumerate}
\vfill
\end{frame}

%==============================================================================
\begin{frame}
  \frametitle{16進数(問題2/4)}
\emph{問題5:}次の16進数を8桁の2進数に変換しなさい.
\begin{enumerate}
\item[1)] $11_{16}$
\vfill
\item[2)] $56_{16}$
\vfill
\item[3)] $AB_{16}$
\vfill
\item[4)] $CD_{16}$
\vfill
\item[5)] $3C_{16}$
\vfill
\end{enumerate}
\end{frame}

%==============================================================================
\begin{frame}
  \frametitle{16進数(問題3/4)}
\emph{問題6:}次の16進数を10進数に変換しなさい.
\begin{enumerate}
\item[1)] $11_{16}$
\vfill
\item[2)] $56_{16}$
\vfill
\item[3)] $AB_{16}$
\vfill
\item[4)] $CD_{16}$
\vfill
\item[5)] $3C_{16}$
\vfill
\end{enumerate}
\end{frame}

%==============================================================================
\begin{frame}
  \frametitle{16進数(問題4/4)}
\emph{問題7:}次の10進数を2桁の16進数に変換しなさい.
\begin{enumerate}
\item[1)] $16_{10}$
\vfill
\item[2)] $50_{10}$
\vfill
\item[3)] $100_{10}$
\vfill
\item[4)] $127_{10}$
\vfill
\item[5)] $130_{10}$
\vfill
\end{enumerate}
\end{frame}

\end{document}
